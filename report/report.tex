% Header
\documentclass[a4paper]{article}%      autres choix : book, report
\usepackage[utf8]{inputenc}%           gestion des accents (source)
\usepackage[T1]{fontenc}%              gestion des accents (PDF)
\usepackage[francais]{babel}%          gestion du français
\usepackage{textcomp}%                 caractères additionnels
\usepackage{mathtools,amssymb,amsthm}% packages de l'AMS + mathtools
\usepackage{lmodern}%                  police de caractère
\usepackage[top=2cm,left=2cm,right=2cm,bottom=2cm]{geometry}%     gestion des marges
\usepackage{graphicx}%                 gestion des images
\usepackage{array}%                    gestion améliorée des tableaux
\usepackage{calc}%                     syntaxe naturelle pour les calculs
\usepackage{titlesec}%                 pour les sections
\usepackage{titletoc}%                 pour la table des matières
\usepackage{fancyhdr}%                 pour les en-têtes
\usepackage{titling}%                  pour le titre
\usepackage{enumitem}%                 pour les listes numérotées
\usepackage{hyperref}%                 gestion des hyperliens
\usepackage{bibentry}

\nobibliography*

\hypersetup{pdfstartview=XYZ}%         zoom par défaut

\setlength{\droptitle}{-5em}   % This is your set screw
\title{\vspace{1.5cm}Projet: Lustre Model Checker}
\author{Mickaël LAURENT}
\date{\vspace{-5ex}}

\pagenumbering{gobble}

\begin{document}

    \maketitle

    \section{Les choix techniques}
		
	\begin{itemize}
		\item \textbf{Méthode générale}: Le fonctionnement général du model-checker est le suivant.
		On commence par fixer k=0, et on génère les formules pour vérifier si la propriété à montrer est k-inductive ou non, comme dans l'énoncé.
		Si le cas de base n'est pas valide, alors on sait que la propriété est fausse (on peut donc s'arrêter là). Sinon, on vérifie le cas inductif.
		Si il est valide lui aussi, alors la propriété est k-inductive et on peut également s'arrêter là. Sinon, on ne peut rien conclure pour le moment
		(on ne sait pas si c'est la propriété qui est fausse ou si c'est notre hypothèse d'induction qui n'est pas assez forte). Dans ce cas, on recommence avec k=1, et ainsi de suite.
		On s'arrête arbitrairement après k=3, seuil après lequel le model checker admet sa toute impuissance face à la complexité infinie des mathématiques.\\
		\item \textbf{Alt-Ergo Zero}: Connaissant déjà Z3 (je l'avais beaucoup utilisé en stage de M1), j'ai décidé d'utiliser Alt-Ergo Zero pour ce projet,
		afin d'étendre mon expérience et de pouvoir comparer les deux SMT-solvers. Alt-Ergo Zero s'est trouvé être particulièrement adapté à ce projet:
		il est très minimal et reconnait exactement le fragment de logique dont j'ai eu besoin. Il est ainsi très adapté aux besoins et ne possède rien superflu).
		J'ai cependant eu quelques difficultés avec ce SMT-solver: voir section 2 et 3.\\
		\item \textbf{Noms de variable}: Pour garantir l'unicité des noms de variable (car une même variable du code peut donner lieu à plusieurs instances,
		par exemple si elle est locale à un noeud qui est appelé plisuers fois), j'ai décidé d'utiliser un identifiant unique de la forme "NAME\_\_ID\_\_NONCE"
		où NAME est le nom dela variable (utile pour déboguer), ID son identifiant dans l'AST, et NONCE un identifiant unique (incrémenté pour chaque nouvelle variable).\\
		\item \textbf{La gestion des flottants}: Alt-Ergo Zero ne gère pas les flottants, mais les fractions rationnelles (module Num de OCaml).
		J'ai donc dû convertir chaque valeur flottante ('real') du programme en fraction. Pour cela, j'ai implémenté une fonction qui converti un flottant en une fraction
		deucimal irréductible (on peut facilement changer la base et utiliser des fractions décimales par exemple, mais cela est encore plus éloigné de la représentation réelle des flottants).
		Cette conversion peut engendrer une (très) grande complexité si le flottant en question est très petit ou très grand (par exemple de l'ordre de $2^{-100}$),
		c'est pourquoi j'ai spécifié une valeur de dénominateur maximale à ne pas dépasser (le flottant sera approximé dans ce cas).
		Même lorsque le flottant est converti de manière exacte, son comportement n'est pas exactement le même que celui du Num qui le représente
		(les flottants sont des approximations et donc risquent d'accumuler des erreurs au fil du temps).
		Ainsi, notre model-checker n'est pas totalement fiable lorsque des flottants ('real') sont utilisés (il peut, dans certains cas, considérer comme vrai
		un programme qui ne l'est pas du fait du caractère approximatif des flottants). J'ai écris un nouvel exemple (ex005.lus) qui teste le model checker sur les flottants.
	\end{itemize}

	\section{Difficultés rencontrées et limitations}

	\begin{itemize}
		\item \textbf{Utilisation d'Alt-Ergo Zero}: J'ai utilisé uniquement Alt-Ergo Zero (pas d'interface avec d'autres SMT-solver tels que Z3).
		Bien que minimal et adapté aux besoins, ce SMT-solver m'a tout de même posé quelques difficultés détaillées ci-après.\\
		\item \textbf{Distinction terme/formule}: Alt-Ergo Zero fait une distinction entre les termes (avec lesquels on peut representer les entiers
		et utiliser l'arithmétique) et les formules (booléenes, avec lesquels on peut utiliser les opérateurs booléens).
		Cette distinction n'existe pas dans le code, où le type booléen est un type comme un autre.
		Heureusement, il est facile de convertir un terme en formule (en le comparant avec le terme 't\_true') et inversement (grâce à la construction 'make\_ite').\\
		\item \textbf{Fonctions de conversion}: Les fonctions de conversion entier/flottant (de type TE\_Prim dans l'AST) m'ont posées quelques problèmes.
		Je n'ai pas trouvé de fonction pour changer le type d'un terme dans Alt-Ergo Zero, j'ai donc encodé cela en introduisant une nouvelle variable du type voulu
		et en ajoutant des (in)équations pour contraindre sa valeur (par exemple, pour convertir un flottant $x$ en entier, on introduit une nouvelle
		variable entière $n$ et on ajoute les inégalités $n <= x$ et $x < n+1$). Cependant, cela semble être source de nombreuses difficultés pour Alt-Ergo Zero:
		lors de tests utilisant ces conversions, Alt-Ergo Zero m'indiquait souvent que le système était 'Unsolvable'.
		J'ai donc décidé d'utiliser ces conversions uniquement lorsque l'utilisateur fait un appel explicite à int\_of\_real ou real\_of\_int, et donc de ne pas introduire
		de nouvelles conversions dans le code, même lorsque cela aurait été préférable (voir par exemple le cas de la division entière dans le code).\\
		\item \textbf{Pas de substitution possible}: J'ai initialement pensé à construire la formule logique décrivant le système pour une variable $n$ générique,
		puis à substituer cette variable $n$ par ce dont j'ai besoin par la suite ($0$, $1$\dots pour l'initialisation, $n$, $n+1$ \ldots pour l'induction).
		Cependant, une fois une formule construite avec Alt-Ergo Zero, il n'est pas possible de substituer une variable (ou du moins je n'ai pas trouvé comment faire).
		Une solution aurait été d'introduire un AST intermédiaire pour les formules qui gère les substitions, puis de les convertir en formule de Alt-Ergo Zero par la suite.
		Cette solution aurait également permit de facilement intégrer plusieurs SMT-solvers (la seule partie à coder pour chaque SMT-solver serait la traduction de l'AST intermédiaire
		en formule spécifique au SMT-solver). Cependant, comme je n'avais pas dans l'optique d'utiliser plusieurs SMT-solvers, j'ai opté pour une solution plus simple:
		recalculer les formules à chaque fois, avec la valeur de $n$ désirée. Cela peut sembler peu efficace, mais cette opération est de toute manière très rapide
		(quasiment linéaire), et a donc un impact négligeable sur les performances (la résolution du système par le SMT-solver est beaucoup plus coûteuse).\\
		\item \textbf{Débogage nécessaire}: J'ai dû faire face à deux problèmes d'exécution d'Alt-Ergo Zero (exceptions non rattrapées en runtime). Les exceptions en question n'étant pas
		très évocatrices, j'ai dû déboguer le programme pour voir quelles en étaient les causes. Je détaille cela dans la prochaine section.
	\end{itemize}

	\section{Modifications d'Alt-Ergo Zero}

	\begin{itemize}
		\item \textbf{Conjonctions/disjonctions}: La première exception à laquelle j'ai été confronté était dû à mon utilisation des combinateurs
		'And' et 'Or'. En effet, bien que l'on puisse les utiliser avec un nombre quelconque de formules directement
		(ce ne sont pas des opérateurs binaires mais à priori des opérateurs d'arité quelconque),
		Alt-Ergo Zero échouera à l'exécution si l'on utilise le combinateur Or avec moins de deux formules
		(cela peut aussi arriver avec un opérateur And car il peut se transformer en Or après une négation).
		Cela est dû à un cas non géré dans la fonction 'mk\_cnf' (la transformation de formules en sform vers des clauses cnf).
		Ne connaissant pas bien le fonctionnement d'Alt-Ergo Zero, j'ai préféré modifier mon algorithme afin d'éviter ce cas pathologique,
		plûtot que de modifier le code d'Alt-Ergo Zero.\\
		\item \textbf{Sauvegarde/restauration des environments}: Le second problème que j'ai rencontré s'est trouvé être
		lié, après débogage, au système de restauration des environments d'Alt-Ergo Zero (utilisé pour le backtracking).
		En effet, la méthode 'cancel\_until' échouait parfois. Cela semble être dû au module Vec utilisé, qui, lors de la récupération d'un élément,
		retourne l'exception Not\_found au lieu de l'élément si ce dernier est égal à une certaine valeur par défaut ('dummy').
		Lorsque l'environment à restaurer est l'environment vide (je ne sais pas si cela est censé arriver, mais en tout cas ça arrive sur la plupart des formules que je génère),
		alors l'appel à la méthode 'get' du module Vec lance l'exception 'Not\_found' qui n'est pas rattrapée.
		J'ai modifié ce comportement dans le module Vec (afin de retourner l'élément même s'il est égal à l'élément par défaut,
		ce qui semble plus cohérent par rapport au comportement de certaines autres méthodes du module Vec).
		Je ne suis pas totalement sûr que ce changement soit sans effet néfaste à d'autres endroits du code, mais experimentalement cela m'a permis de régler
		le problème et donne des résultats exactes sur les quelques tests que j'ai fait.
		Vous pouvez tester mon programme sans ce correctif en recompilant une version non modifiée d'AEZ, mais dans ce cas le model checker devrait échouer sur la plupart des exemples.
	\end{itemize}
		
\end{document}